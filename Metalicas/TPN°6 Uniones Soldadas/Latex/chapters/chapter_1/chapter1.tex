\begin{center}
\underline{\Large{T.P.N°6: Uniones soldadas y elementos sometidos a tracción}}
\end{center}

\begin{enumerate}
\item Diseñar la unión soldada viga – columna. La viga es un perfil IPN 200 y la columna está
formada por dos perfiles UPN 200. Tiene una carga aplicada de 40kN con una excentricidad
de 1m como se muestra en la figura. El acero de la perfileria es F-24 y el la resistencia del
material de aporte del electrodo tiene una resistencia mínima a tracción de 480MPa.

\begin{figure}[H]
\begin{center}
     \includegraphics[scale = 1]{chapters/chapter_1/images/figura1.png}
\end{center}
\caption{Unión soldada viga – columna}
\end{figure}
\item Redimensionar la unión del ejercicio 1 del trabajo práctico N° 5, utilizando soldadura. El
material de aporte del electrodo tiene una resistencia mínima a tracción de 480MPa.

\item Verificar la barra a tracción de del ejercicio 1 del trabajo práctico N° 5, tanto para la unión
abulonada, como soldada.

\end{enumerate}

\newpage

\begin{center}
\underline{\Large{Solución}}
\end{center}

\begin{enumerate}
\item Unión soldada viga – columna. La viga es un perfil IPN 200 y la columna está
formada por dos perfiles UPN 200.
\begin{itemize}
\item \underline{Datos}
\begin{align*}
& \text{Acero F-24}\\
& F_u = 370MPa\\
& F_y = 235MPa
\end{align*}

\begin{figure}[H]
\begin{center}
     \includegraphics[scale = 1]{chapters/chapter_1/images/figura2.png}
\end{center}
\caption{Perfiles IPN200 y UPN200}
\end{figure}

\begin{align*}
& \text{IPN200}\\
& t_1 = 0.75 cm
\end{align*}
\begin{align*}
& \text{UPN200}\\
& t_2 = 0.85 cm
\end{align*}
\begin{align*}
& \text{Electrodo}\\
& F_u = 480MPa\\
\end{align*}


\item \underline{Distancias y separaciones}
	\begin{itemize}
	\item Distancias mínimas al borde:\\
	Según la tabla J.3.4 para bulones de 3/4 y bordes laminados $d_{borde} = 26 mm$
	\item Separación mínima entre bulones:\\
	$s_{min} = 3 \cdot d = 3 \cdot 1.90cm = 5.70cm$
	\item Separación máxima entre bulones:\\
	Para barras no pintadas de acero resistente a la corrosión atmosférica se debe cumplir:
	\begin{align*}
	& s_{max} \leq 14 \cdot t_{min}\\
	& s_{max} \leq 14 \cdot 0.75cm = 10.5cm\\
	& s_{max} \leq 180mm
	\end{align*}
	\end{itemize}

\end{itemize}
\item Redimensionar la unión del ejercicio 1 del TPN° 5.



\item Verificar la barra a tracción del ejercicio 1 del TPN° 5.


\end{enumerate}