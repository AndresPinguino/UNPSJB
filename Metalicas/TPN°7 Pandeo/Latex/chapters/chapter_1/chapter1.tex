\begin{center}
\underline{\Large{T.P.N°7: Pandeo}}
\end{center}

\begin{enumerate}
\item Se tiene una columna compuesta por perfiles UPN de 35 cm x 60cm, donde se aplica
una carga de 1000 kN en el baricentro de la sección. La altura es de 5m y se
considera como condición de vínculo empotrado - articulado en el plano YZ y
empotrado libre en el plano XZ. Uno de los cordones está formado por 2UPN 160
(unidos por diagonales) y el otro por dos UPN 140 (unidos por diagonales). Ambos
cordones se encuentran unidos mediante diagonales y montantes como se indican en
el dibujo. El acero a utilizar es F-24. Se requiere:
\begin{itemize}
\item Verificar la columna, en caso de no ser así, redimensionar
\item Dimensionar los elementos de enlace
\end{itemize}

\begin{figure}[H]
\begin{center}
     \includegraphics[scale = 1]{chapters/chapter_1/images/figura1.png}
\end{center}
\caption{Columna compuesta por perfiles UPN de 35 cm x 60cm}
\end{figure}
\item Determinar la carga ultima de una columna formada por 4 perfiles ángulos de 3 ½” x
½”. La altura es de 6m y se considera como condición de vínculo empotrado -
articulado en el plano YZ y empotrado libre en el plano XZ. La unión en el plano
YZ son mediante diagonales y la unión en el plano XZ son mediante diagonales y
montantes. El acero a utilizar es F-24. Dimensionar los elementos de enlaces.

\end{enumerate}

\newpage

\begin{center}
\underline{\Large{Solución}}
\end{center}

\begin{enumerate}
\item Unión soldada viga – columna. La viga es un perfil IPN 200 y la columna está
formada por dos perfiles UPN 180.
\begin{itemize}
\item \underline{Datos}
\begin{align*}
& \text{Acero F-24}\\
& F_u = 370MPa\\
& F_y = 235MPa
\end{align*}

\begin{figure}[H]
\begin{center}
     \includegraphics[scale = 0.8]{chapters/chapter_1/images/figura2.png}
\end{center}
\caption{Perfiles IPN200 y UPN180}
\end{figure}

\item \underline{Estado de Cargas}
\begin{align*}
& P = 40KN \\
& M = P \cdot d = 40KN \cdot 100cm = \framebox{$4000 KN \cdot cm$}
\end{align*}

\item \underline{Estudio del Punto A}
\begin{align*}
& L = 2 \cdot 15.9cm = \framebox{$31.8cm$} \\
& A_w = 0.707 \cdot W \cdot L \Rightarrow \text{área efectiva} \\
& A_w = 0.707 \cdot 0.6 cm \cdot 31.8cm = \framebox{$13.48 cm^2$} \\
\end{align*}

\end{itemize}
\end{enumerate}